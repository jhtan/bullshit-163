\begin{tabular}{p{0.4\textwidth}p{0.5\textwidth}}
\textbf{título del proyecto} & Sistema de movilidad, registro y retribución pra el hospital de clínicas La Paz \\
\textbf{nombre del tutor} & Luisa Velasquez Lopez \\
\textbf{nombre del revisor} & Franz Cuevas Quiroz \\
\textbf{fecha de presentación} & 2011
\end{tabular}

\begin{enumerate}[a]
\item Programación Extrema
\item Desarrollar un sistema web para el control de activos fijos y almacenes para la Fundación Cuerpo de Cristo en El Alto.
\item \textbf{Objetivo General} Diseñar e implementar un sistema de información para la unidad de recursos humanos del hospital de clínicas de la ciudad de La Paz que permita gestionar fácilmente la información de su capital humano, mejorando los procesos de movilidad, registro y retribución. Proporcionando de esta forma información confiable, precisa y oportuna a nivel estratégico y operativo. \\
\textbf{Objetivos Específicos} Automatizar los proceso de movilidad, registro y retribución, diseñar e implementar un módulo de registro de datos que permita integrar y actualizar la información obtenida de los recursos humanos, diseñoar e implementar un modulo de movildadd de personas que permita el control y seguimiento de los cambios a los que esta sueto el personal desde su ingreso hasta su retiro de la institución, diseñar e implementar un modulo de retribución de personal que gestione e l pago de haberes de forma ágil, sencilla y transparente. Generar estadísticas e información sobre las pirniicipales caracteristicas de los recursos humanos de l sinstitución que sirvan como herramientas de control y seguimientos para la toma de decisiones, incrementar el control sobre los gastos de salarios, a partir de un valor estimado o pronosticaso obtenido a través de u modelo matemático utilizando dat os estadísticos de periosos anteriores.

\item \textbf{Lista de Requerimientos} \begin{itemize}
  \item Automatizar los proceso de movilidad, registro y retribución,
  \item Diseñar e implementar un módulo de registro de datos que permita integrar y actualizar la información obtenida de los recursos humanos,
  \item Diseñoar e implementar un modulo de movildadd de personas que permita el control y seguimiento de los cambios a los que esta sueto el personal desde su ingreso hasta su retiro de la institución,
  \item Diseñar e implementar un modulo de retribución de personal que gestione e l pago de haberes de forma ágil, sencilla y transparente.
  \item Generar estadísticas e información sobre las pirniicipales caracteristicas de los recursos humanos de l sinstitución que sirvan como herramientas de control y seguimientos para la toma de decisiones,
  \item incrementar el control sobre los gastos de salarios, a partir de un valor estimado o pronosticaso obtenido a través de u modelo matemático utilizando dat os estadísticos de periosos anteriores.
\end{itemize}


\end{enumerate}

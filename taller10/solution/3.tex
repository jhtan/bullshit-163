\textit{It computes the distance (smallest number of edges) from $s$ to each reachable vertex.}\footnote{Introduction to Algorithms, Thomas H. Cormen, Charles E. Leiserson, Ronald L. Rivest, Clifford Stein, Breadth-first search topic, page 594}

En este documento de berkeley\footnote{\url{https://www.cs.berkeley.edu/~vazirani/algorithms/chap4.pdf}}, definen la distancia entre un nodo y otro, como la ``longitud del camino más corto entre un nodo y otro''. Si se continúa la lectura, se puede ver que también asumen que la ``longitud de un camino'' es la cantidad de vértices por el que pasa (no así, la cantidad de nodos).

\textbf{Por definición, la longitud de un camino en teoría de grafos, es la cantidad de vértices que atravieza.}

\begin{itemize}
\item Tamaño
  \begin{align*}
    21 + 20 = 41
  \end{align*}

\item Profundidad
  \begin{align*}
    3
  \end{align*}

\item Amplitud
  \begin{align*}
    12
  \end{align*}

\item Relación arco-nodo
  \begin{align*}
    0.95
  \end{align*}
\end{itemize}

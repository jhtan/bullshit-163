\textbf{Basado en Componentes y Prototyping}

Los docentes necesitan probar constantemente el nuevo sistema, para sentirse cómodos, para que este nuevo sistema no caiga en el desuso (como la página de la carrera de informática). Por esta razón es importante el \textit{prototyping}

El desarrollo tal vez no sea urgente, pero lo más fundamental puede ser desarrollado de forma rápida, de modo que el sistema entre en funcionamiento lo más pronto posible, y se le vayan añadiendo características \textit{on the go}.

Es por eso que es necesario el desarrollo \textit{basado en componentes}.

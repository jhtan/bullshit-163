\begin{enumerate}
  \item \textbf{El software no se gasta.} \\
    Viéndolo desde el punto de vista físico el software llegaría a ser un conjunto de bits almacenados en un aparato electrónico, desde este punto de vista obviamente un montón de bits no se podrían gastar. Pero si analizamos el software desde un punto de vista más holístico, como un sistema que cambia la forma de trabajo de las personas, sí se puede decir que se gasta, ya que al pasar el tiempo un proceso puede llegar a ser innecesario por el cambio del entorno.
  \item \textbf{El cambio es inevitable.} \\
    Algunos cambios pueden ser evitables, pero por lo general no. Por ejemplo si uno puede decidir en realizar un cambio o no, podría decidir en no realizar el camio si este tiene efectos económicos muy severos. Pero si el cambio es estructural y todo el sistema falla si no se hace este, es inevitable.
  \item \textbf{El mantenimiento mejora el software.} \\
    Si el software que se mantiene está desarrollado con buenas prácticas de programación y cumple con la propiedad de escalabilidad, se podría decir que si se está mejorando este cuando en el proceso de mantenimiento se le hace una mejora o se repara un error. Pero en lo general en el mantenimiento el software se va deteriorando, ya que no todo el software está bién hecho y los nuevos programadores no saben en qué asumpsiones quedaron los desarrolladores originales del proyecto.
  \item \textbf{El costo real de la posesión del software es bajo.} \\
    Obviamente eso depende del software. En todo caso es mucho mas caro si el cliente también quiere poseer el código fuente.
  \item \textbf{El software nunca muere.} \\
    Si podría morir si el proceso del sistema cambia. También si se encuentra muy deteriorado y ``parchado''.
  \item \textbf{El tamaño no importa.} \\
    Eso me parece relativo. Pero si hablamos estrictamente de software, si el software es grande, el código también lo será, lo que implica que se necesitará mas tiempo para hacerlo, eso significa que se deben contratar más personas y se gastará mas dinero. Así sí importaría.
\end{enumerate}
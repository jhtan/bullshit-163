\textbf{Programación Extrema y Prototyping}

Los cursos de verano están cerca, y se debe crear un sistema sólido en 6 meses. Es importante que el software creado sea seguro y sólido, para no repetir el fiasco del sia.

Probablemente el sistema creado para los cursos de verano, será constantemente atacado por hackers y lamers, así que el sistema tiene que ser capaz de soportarlo.

La \textit{programación extrema} nos ayudará a terminar el producto en el menor tiempo posible, para que esté listo para estos cursos de verano.

Si en producción el sistema es atacado y requiere cambios extremos, esta metodología es perfecta.

Los estudiantes deben sentirse cómodos con el sistema, si al jefe de carrera le interesa construir una buena reputación.

Por eso es importante usar \textit{prototyping}, para mostrar a la comunidad estudiantil constantemente un prototipo, para que se sientan seguros, para que recuperen la confianza en las autoridades, y así hacerles conocer que no se está perdiendo el tiempo.

Además, esto proveerá retroalimentación sobre posibles mejoras al sistema, y talvez hasta sobre sus fallas de seguridad.

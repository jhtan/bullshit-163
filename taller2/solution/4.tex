\textbf{Prototyping y Spiral}

El sistema tendría un carácter no muy crítico. No es necesario que esté listo en el primer release. La forma en la que el sistema será usado, permite que hayan cambios constantemente, y provee flexibilidad en su desarrollo.

El modelo de \textit{Prototyping}, reduce el riesgo inherente de los proyectos grandes, ya que lo parte en segmentos más pequeños, y da flexibilidad, de modo que sea más fácil realizar cambios \textit{on the go}. Además, permite lanzar prototipos para que los \textit{stakeholders} prueben el producto.

Como cada uno de los ciclos del modelo \textit{Spiral} comienza por identificar la retroalimentación de los \textit{stakeholders}; esta retroalimentación, permitirá que el producto final sea mejor en cada iteración.

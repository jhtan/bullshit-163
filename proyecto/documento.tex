\documentclass{article}
\usepackage[spanish]{babel}
\usepackage[utf8]{inputenc}

\usepackage{multicol}
\usepackage{mathtools}
\usepackage{amssymb}
\usepackage{amsmath}
\usepackage{amsfonts}
\usepackage{graphics}
\usepackage{listings}
\usepackage{listingsutf8}
\usepackage{setspace}
\usepackage{xcolor}
\lstset{inputencoding=utf8/latin1}

%\usepackage[margin=1cm]{geometry}
\setlength{\parskip}{0.5cm plus4mm minus3mm}
%\setlength{\parindent}{0pt}

%\renewcommand*{\familydefault}{\sfdefault}


\begin{document}
\begin{section}{Documento de Requisitos}
Tema 3: ingenieria de requerimientos.pdf, página 18.
\end{section}

\begin{section}{Casos de uso}
Tema 4: requerimientos con UML (dice que también sirve para diseño procedimental).
\end{section}

%% Tema 7: diseño con UML. página 35
\begin{section}{Análisis}
  \begin{subsection}{Diagrama de secuencias}
  \end{subsection}

  \begin{subsection}{Diagrama de clases conceptuales}  
  \end{subsection}

  \begin{subsection}{Diagrama de clases de análisis}
  \end{subsection}
\end{section}

\begin{section}{Diseño}
  \begin{subsection}{Diagrama de clases y objetos}
  \end{subsection}

  \begin{subsection}{Diagramas de colaboración}
  \end{subsection}

  \begin{subsection}{Diagramas de secuencia}
  \end{subsection}

  \begin{subsection}{Diagramas de estados.}
  \end{subsection}
\end{section}


\end{document}

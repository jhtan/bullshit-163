\documentclass{article}
\usepackage[spanish]{babel}
\usepackage[utf8]{inputenc}

\usepackage{enumerate}
\usepackage{forloop}
\usepackage{multicol}
\usepackage{mathtools}
\usepackage{amssymb}
\usepackage{amsmath}
\usepackage{amsfonts}
\usepackage{graphics}
\usepackage{listings}
\usepackage{listingsutf8}
\usepackage{setspace}
\usepackage{xcolor}
\lstset{inputencoding=utf8/latin1}

%\usepackage[margin=1cm]{geometry}
\setlength{\parskip}{0.5cm plus4mm minus3mm}
%\setlength{\parindent}{0pt}

%\renewcommand*{\familydefault}{\sfdefault}
\newcommand{\numberOfProblems}{7}

\begin{document}

\title{Taller 4: Problemas de Requerimientos con UML}
\date{\today}
\author{Jhonatan I. Castro Rocabado \\ Ruth Margarita García López}

\maketitle

\newcommand{\solutionBox} [1] {
  \begin{center}
    \colorbox{blue!20}{
      \begin{minipage}[\textheight]{\textwidth}
        \textsc{Solución} \\
        #1
      \end{minipage}
    }
  \end{center}
}

\begin{enumerate}
  \newcounter{i}
  \forloop{i}{1}{\value{i} < \numexpr\numberOfProblems+1\relax}{
  \item \input{statement/\arabic{i}} \\

    \solutionBox{
      \IfFileExists{solution/\arabic{i}}{
        \input{solution/\arabic{i}}
      }{
        I accidentally the whole solution :(
      }
    }
  }
\end{enumerate}

\end{document}

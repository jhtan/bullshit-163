Desarrolle los casos de uso para un sistema de encuentros virtuales (parecido a un chat) con la siguiente descripción: (1) Cuando se conecta al servidor, un usuario puede entrar o salir de un encuentro. (2) Cada encuentro tiene un administrador. (3) El administrador es el usuario que ha planificado el encuentro (el nombre del encuentro, la agenda del encuentro y el moderador del encuentro). (4) Cada encuentro puede tener también un moderador designado por el manager. (5) La misión del moderador es asignar los turnos de palabra para que los usuarios hablen. (6) El moderador también podrá dar por concluido el encuentro en cualquier momento. (7) En cualquier momento un usuario puede consultar el estado del sistema, por ejemplo los encuentros planeados y su información.

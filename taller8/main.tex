\documentclass{article}
\usepackage[spanish]{babel}
\usepackage[utf8]{inputenc}

\usepackage{url}
\usepackage{enumerate}
\usepackage{forloop}
\usepackage{multicol}
\usepackage{mathtools}
\usepackage{amssymb}
\usepackage{amsmath}
\usepackage{amsfonts}
\usepackage{graphics}
\usepackage{listings}
\usepackage{listingsutf8}
\usepackage{setspace}
\usepackage{xcolor}
\lstset{
  inputencoding=utf8/latin1,
  language=Java,
  numbers=left,
  basicstyle=\footnotesize,
  frame=shadowbox,
  rulesepcolor=\color{gray},
  breaklines=true,
  title=\lstname
}

%\usepackage[margin=1cm]{geometry}
\setlength{\parskip}{0.5cm plus4mm minus3mm}
%\setlength{\parindent}{0pt}

%\renewcommand*{\familydefault}{\sfdefault}
\newcommand{\numberOfProblems}{3}

\begin{document}

\title{Taller 8: Problemas de programación Java}
\date{\today}
\author{Jhonatan I. Castro Rocabado \\ Ruth Margarita García López}

\maketitle

\newcommand{\solutionBox} [1] {
  #1
}

\begin{enumerate}
  \newcounter{i}
  \forloop{i}{1}{\value{i} < \numexpr\numberOfProblems+1\relax}{
  \item \input{statement/\arabic{i}}

    \solutionBox{
      \IfFileExists{solution/\arabic{i}}{
        \input{solution/\arabic{i}}
      }{
        I accidentally the whole solution :(
      }
    }
  }
\end{enumerate}

\end{document}

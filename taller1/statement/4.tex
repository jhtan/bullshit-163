La historia ha conservado pocos rasgos biográficos de Diofanto, notable matemático de la antigüedad. Todo lo que se conoce acerca de él ha sido tomado de la dedicatoria que figura en su sepulcro, inscripción compuesta en forma de ejercicio matemático. Se reproduce esta inscripción: “¡Caminante! Aquí fueron sepultados los restos de Diofanto. Y los números pueden mostrar, ¡oh, milagro!, cuán larga fue su vida, cuya sexta parte constituyó su hermosa infancia. Había transcurrido además una duodécima parte de su vida, cuando de vello cubrióse su barbilla y la séptima parte de su existencia transcurrió en un matrimonio estéril. Pasó un quinquenio más y le hizo dichoso el nacimiento de su precioso primogénito, que entregó su cuerpo, su hermosa existencia, a la tierra, que duró tan sólo la mitad de la de su padre. Y con profunda pena descendió a la sepultura, habiendo sobrevivido cuatro años al deceso de su hijo”. ¿Cuántos años había vivido Diofanto cuando le llegó la muerte?

Lo primero que haremos será representar los datos en ecuaciones, {m} será la carga del mulo y {c} la carga del caballo:

\begin{equation*}
  m+1 = 2(c-1)
\end{equation*}

\begin{equation*}
  m-1 = c+1
\end{equation*}

Ahora igualaremos ambas ecuaciones a {m}:

\begin{equation*}
  m = 2c-3
\end{equation*}

\begin{equation*}
  m = c+2
\end{equation*}

Igualando las ecuaciones y despejando:

\begin{equation*}
  2c-3 = c+2
\end{equation*}

Finalmente como soluciones tenemos:

\begin{equation*}
  c = 5
\end{equation*}

\begin{equation*}
  m = 7
\end{equation*}

La solución es que el mulo tiene una carga de peso 7 y el caballo una de peso 5.

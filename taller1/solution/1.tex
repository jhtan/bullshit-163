Este es un problema clásico de movimiento de partículas en el que se trabaja con velocidades uniformemete variantes, así que utilizaremos la siguiente fórmula:

\begin{equation*}
  V = \frac{d}{t}
\end{equation*}

Sean las velocidades del obrero jóven y el obrero viejo respectivamente:

\begin{equation*}
  V_J = \frac{x}{t+5}, V_V = \frac{x}{t}
\end{equation*}

Luego despejando la distancia {x} en la que se encontrarán ambas personas y colocando sus respectivas velocidades:

\begin{equation*}
  x = \frac{20}{d}(t+5), x = \frac{30}{d}(t)
\end{equation*}

Finalmente igualando ambas ecuaciones:

\begin{equation*}
  \frac{20}{d}(t)+\frac{100}{d} = \frac{30}{d}(t)
\end{equation*}
\begin{equation*}
  \frac{100}{d} = \frac{10}{d}(t)
\end{equation*}

La solución sería:

\begin{equation*}
  t = 10 min
\end{equation*}

Conclusión: El obrero jóven se encuentra con el obrero viejo en el minuto 10.
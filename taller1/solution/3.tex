El dinero de todos, suma 45

\begin{align*}
  a + b +c+d=45
\end{align*}

Pero si aplicamos ciertas condiciones, todos tendrán la misma cantidad de dinero

\begin{align*}
  a+2 = b-2 = 2c = \frac{d}{2}
\end{align*}

Ponemos a todos en términos de $a$:

\begin{align*}
  b&=a+4 \\
  c&=\frac{a+2}{2} \\
  d&=2(a+2)
\end{align*}

Y reemplazamos en la primera ecuación

\begin{align*}
  a + a+4+\frac{a+2}{2}+2(a+2) &= 45 \\
  \frac{9a}{2} + 9 &= 45 \\
  a &=8
\end{align*}

y reemplazamos en el resto de ecuaciones

\begin{align*}
  b = 8+4 &= 12 \\
  c = \frac{8+2}{2} &= 5 \\
  d = 16 + 4 &= 20
\end{align*}

Entonces, los hermanos tienen: 8, 12, 5, 20 pesos bolivianos, respectivamente
